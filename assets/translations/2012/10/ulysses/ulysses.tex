% Default Compiler: txs:///xelatex
% Default Bibliography Tool: BibTex

\documentclass[12pt,onecolumn,a4paper]{book}
\usepackage{epsfig,graphicx,subfigure,amsthm,amsmath}
\usepackage{caption}
\usepackage{color,xcolor}
\usepackage[linktocpage=true,colorlinks,citecolor=blue,pagebackref=true]{hyperref}
\usepackage[top=30mm, bottom=30mm, left=30mm, right=30mm]{geometry}
\usepackage[T1]{fontenc}
\usepackage[utf8]{inputenc}
\newcommand{\noun}[1]{\textit{\textcolor{gray}{#1}}}

\usepackage{authblk}
\usepackage{polyglossia}
\setotherlanguage{english}
\setotherlanguage{persian}
\usepackage[nonamebreak]{natbib}
\usepackage{xepersian}
\settextfont[Scale=1.2]{IRLotus}
\defpersianfont\mfo[Scale=1.2]{IRLotus}
\setlatintextfont[Scale=1]{Doulos SIL}

\begin{document}
    \title{خلاصۀ اولیس \LTRfootnote{\lr{SparkNotes Editors. ``SparkNote on Ulysses.'' SparkNotes.com. SparkNotes LLC. 2003.}}}
    \author{جیمز جویس\\
    ترجمه وحید مواجی
    }
    \date{مهر 1391}
    \frontmatter                            % only in book class (roman page #s)
    \maketitle                              % Print title page.
    \tableofcontents                        % Print table of contents
    \mainmatter


    \part{}
    \chapter{شخصیت‌ها}
    \noun{لئوپلد بلوم}\LTRfootnote{\lr{Leopold Bloom}} - مردی سی و هشت ساله و مسئول تبلیغات در دوبلین. \noun{بلوم} در دوبلین با \noun{رودلف}\LTRfootnote{\lr{Rudolph}}، پدر یهودیِ مجارستانی‌اش‌ و \noun{الن}\LTRfootnote{\lr{Ellen}}، مادر کاتولیکِ ایرلندی‌اش بزرگ شده است. او از مطالعه و تفکر دربارۀ علوم و اختراعات و شرح معلوماتش به دیگران لذت می‌برد. \noun{بلوم}، عاطفی و کنجکاو است و عاشق موسیقی می‌باشد. او ذهنش درگیر روابط سردش با زنش \noun{مالی} است.

    \part{}
    \chapter[تلماخوس]{تلماخوس\protect\footnote{\lr{Telemachus}-\rl{ در اسطوره‌های یونان پسر اولیس و پنلوپه است. در کودکی پسری ترسو بود. اما آتنه به او شجاعت بخشید. در دوران سرگردانی پدر به جست‌وجویش رفت.}}}

    \chapter[نستور]{نستور\protect\footnote{\lr{Nestor}-\rl{در اسطوره‌های یونان، شاه پولوس است. پسر نرئوس و خلوریس بود. از میان دوازده پسر نرئوس تنها او بود که از حملۀ هراکلس به پولوس جان سالم در برد. در جنگ تروا سالخورده‌ترین و عاقل‌ترین جنگاور محسوب می‌شد.}}}

    \chapter[پروتئوس]{پروتئوس\protect\footnote{\lr{Proteus}-\rl{در اسطوره‌های یونان، یکی از خدایان دریا که اشکال مختلف به خود می‌گرفته است. زمانی او را پسر پوزئیدون دانسته‌اند و زمانی ملازم و همنشین وی. او هم از قدرت پیشگویی برخوردار بود و هم از قدرت تغییر شكل در هر زمان كه اراده می كرد.}}}

    \chapter[کالوپسو]{کالوپسو\protect\footnote{\lr{Calypso}-\rl{در اسطوره‌های یونان، یک پری دریایی است. دختر اطلس بود و در جزیره اوگوگیا زندگی می‌کرد. اولیس در راه بازگشت از تروا، به این جزیره وارد شد. کالوپسو به او دل بست و او را هفت سال نزد خود نگاه داشت. به اولیس پیشنهاد کرد همیشه با او بماند و جاودان شود. اما اولیس در هوای خانه بود. عاقبت زئوس، هرمس را فرستاد تا کالوپسو را راضی کند دست از اولیس بردارد.}}}

    \chapter[لوتوفاگ‌ها]{لوتوفاگ‌ها\protect\footnote{\lr{Lotus-eaters} یا \lr{Lotophagi} یا \lr{Lotophaguses}-\rl{به معنی خورندگان نیلوفر آبی، در اسطوره‌های یونان، نام قبیله‌ای است که در ساحل لیبی زندگی می‌کردند. اولیس با همراهانش به جزیره لوتوفاگ‌ها وارد شد. آن‌ها از میوه نیلوفر آبی خوردند و حافظه خود را از دست دادند.}}}

    \chapter[هادس]{هادس\protect\footnote{\lr{Hades}-\rl{ در اساطیر یونانی، فرمانروای مردگان و دنیای زیرزمین، فرزند کرونوس و رئا است. او در قرعه‌کشی با برادرانش، بدترین سهم را برنده شد و آن جهان زیرین یا دنیای مردگان بود درصورتی که برادران او زئوس و پوزئیدون به ترتیب آسمان و دریا نصیبشان شد. از آنجایی که رعایای هادس را مردگان تشکیل می‌دادند، او به کسانی که موجب افزایش جمعیت سرزمینش می‌شدند بسیار علاقه داشت. مانند ارینی‌ها Erinnyes یا خشم و ناامیدی، که کارشان تعقیب گناهکاران و سوق دادن آن‌ها به سمت خودکشی بود.}}}

    \chapter[آیولوس]{آیولوس\protect\footnote{\lr{Aeolus}-\rl{ در اسطوره‌های یونان، پادشاه جزیره شناور آیولیا و خدای زمینی بادها است. زئوس قدرت مهار بادها را به او داده بود و خدایی زمینی محسوب می‌شد. با مهار باد به اولیس در رسیدن به تروا کمک کرد.}}}

    \chapter[لاستریگون‌ها]{لاستریگون‌ها\protect\footnote{\lr{Laestrygonians} یا \lr{Laestrygones} یا \lr{Laistrygones}-\rl{ در اساطیر یونان، قبیله‌ای از غول‌های آدمخوار. اولیس در راه بازگشت به ایتاکا با آنها مواجه شد. غول‌ها، بسیاری از مردان اولیس را خوردند و یازده کشتی از دوازه کشتی او را با پرتاب سنگ از بالای تپه نابود کردند.}}}

    \chapter[سیلا و کاریبد]{سیلا و کاریبد\protect\footnote{\lr{Scylla and Charybdis}-\rl{یا اسکیلا و کاریبدس یا سیلا و شاریبدیس، دو هیولای دریایی از اساطیر یونان هستند که توسط هومر مورد اشاره قرار گرفته اند. بعد از سنت‌های یونانی، محل آنها را در دو طرف تنگه مسینا و مقابل یکدیگر، در نظر گرفته‌اند. این محل بین سیسیل و سرزمین اصلی گراسیا مگنا یا همان یونان بزرگ (در جنوب ایتالیا) واقع است. گفته می‌شود که سیلا و کاریبد، در واقع در کنار یکدیگر، تهدیدی جدی و غیر قابل اجتناب در مسیر عبور ملوانان به حساب می‌آمدند؛ بدین ترتیب، کاریبد و سیلا هر دو در خود معنای جلوگیری کننده از عبور را دارند.}}}

    \chapter[صخره‌های سرگردان]{صخره‌های سرگردان\protect\footnote{\lr{Wandering Rocks} یا \lr{Planctae}-\rl{در اساطیر یونان، گروهی از صخره‌ها بودند که دریای بین آنها بیرحمانه خروشان بود.}}}

    \chapter[سیرن‌ها]{سیرن‌ها\protect\footnote{\lr{Sirens}-\rl{یا سایرن، یا حوری دریایی اساطیر یونان، گاهی به صورت موجودی با بدن یک پرنده و سر یک زن، و در سایر موارد به شکل تنها یک زن تصویر شده‌است. سیرن‌ها دختران خدای دریا فورکیس بوده‌اند، هرچند در نسخۀ دیگری از اساطیر، پدرشان خدای نهر، آکلوس دانسته شده است. آنها آوازی بسیار زیبا و فریبنده داشتند و دریانوردان را با آوای خود گمراه کرده و به کام صخره‌های مرگ‌آوری که بر روی آن آواز میخواندند، میکشیدند. اولیس، قهرمان افسانه‌ای یونان، توانست بدون هیچ خطری از جزیره آنان بگذرد، از آنرو که طبق نصیحت سیرسه ساحره، او از همراهانش خواست تا گوشهایشان را با موم پرکرده و او را محکم به دکل کشتی ببندند تا با اغوای آنان کشتی را به بیراهه نکشاند و بی هیچ خطری بتواند آواز آنان را بشنود.}}}

    \chapter[سیکلوپ]{سیکلوپ\protect\footnote{\lr{Cyclops}-\rl{یکی از موجودات افسانه‌ای در اساطیر یونانی است. سیکلوپ‌ها در اسطوره‌های یونان، غول‌هایی با یک چشم در وسط پیشانی هستند. آنها قدرتمند و سرسخت بودند. حرکات آنها همیشه همراه با خشونت و قدرت بود.}}}

    \chapter[ناوسیکائا]{ناوسیکائا\protect\footnote{\lr{Nausicaa}-\rl{در اسطوره‌های یونان، دختر آلکینوئوس است. زمانی که اولیس به جزیرۀ سخریا رسید او عاشق اولیس شد و از پدر خواست اجازه دهد با وی ازدواج کند. اما اولیس که قصد بازگشت به سرزمین خود را داشت نپذیرفت.}}}

    \chapter[گلۀ گاوِ خورشید]{گلۀ گاوِ خورشید\protect\footnote{\lr{Oxen of the Sun} یا \lr{Cattle of Helios}-\rl{در اساطیر یونان، در جزیرۀ تریناکیا می‌چریدند که معادل سیسیل امروزین است. هلیوس یا خدای خورشید، هفت گله گاو و هفت رمه گوسفند داشت.}}}

    \chapter[کیرکه]{کیرکه\protect\footnote{\lr{Circe}-\rl{در اسطوره‌های یونان، دختر هلیوس و پرسه است. جادوگری قدرتمند بود که دشمنانش را به حیوان تبدیل می‌کرد. پیکوس، سکولا و همراهان اولیس از جمله کسانی بودند که مورد خشم او قرار گرفتند.}}}

    \chapter[ائومایوس]{ائومایوس\protect\footnote{\lr{Eumaeus}-\rl{خوک‌چران اولیس. خدمتکار اولیس بود و او را در راه رسیدن به همسرش پنلوپه یاری داد. در اصل شاهزاده بود ولی کنیزی او را ربود و به لائرتس، پدر اولیس فروخت.}}}

    \chapter[ایتاکا]{ایتاکا\protect\footnote{\lr{Ithaca}-\rl{شهری که اولیس در آن زندگی می‌کرد.}}}

    \chapter[پنلوپه]{پنلوپه\protect\footnote{\lr{Penelope}-\rl{در اسطوره‌های یونان، همسر اولیس و مادر تلماخوس است. او در جوانی به دلیل زیبایی‌اش خواستگاران زیادی داشت. پدرش برای جلوگیری از دعوا و کشمکش مسابقه‌ای ترتیب داد تا برندۀ آن را به دامادی انتخاب کند. در این مسابقه اولیس سرافراز بیرون آمد و با پنلوپه ازدواج کرد.}}}

    \part{}
    \chapter{طرح کلی داستان}
    \noun{استیون ددالوس} در حال گذراندن ساعات اولیۀ صبح 16 ژوئن 1904 است و از دوست مسخره‌کننده‌اش، \noun{باک مالیگان} و \noun{هینز}، آشنای انگلیسی \noun{باک} دوری می‌جوید. وقتی \noun{استیون} می‌خواهد سر کار برود، \noun{باک} به او می‌گوید که کلید خانه را با خود نبرد و آنها را در ساعت 12:30 در میخانه ببیند. \noun{استیون} از \noun{باک} می‌رنجد.

    حدود ساعت 10 صبح، \noun{استیون} در کلاس درسش در مدرسۀ پسرانۀ \noun{گرت دیزی} دارد تاریخ درس می‌دهد. بعد از کلاس، \noun{استیون} با \noun{دیزی} ملاقات می‌کند تا حقوقش را بگیرد. \noun{دیزیِ} کوته‌فکر و متعصب، راجع به زندگی برای \noun{استیون} موعظه می‌کند. \noun{استیون} قبول می‌کند که نوشتۀ \noun{دیزی} دربارۀ بیماری احشام را به آشنایانش در روزنامه بدهد.

    \noun{استیون} بقیۀ صبحش را به تنها قدم زدن در ساحل \noun{سندی‌مونت} می‌گذراند و منتقدانه به خودِ جوان‌ترش و به ادراک و الهام فکر می‌کند. او در ذهنش شعری می‌گوید و آن را روی تکه کاغذی که از نوشتۀ \noun{دیزی} پاره کرده است می‌نویسد.

    در ساعت 8 صبح همان روز، \noun{لئوپلد بلوم} مشغول درست کردن صبحانه است و نامه و صبحانۀ زنش را برای او به رختخواب می‌برد. یکی از نامه‌های زنش از طرف مدیر تور کنسرتِ \noun{مالی}، \noun{بلیزس بویلان} است (\noun{بلوم} به این مظنون است که او عاشق زنش نیز هست) – \noun{بویلان} ساعت 4 بعدازظهر امروز قرار ملاقات دارد. \noun{بلوم} به طبقۀ پایین می‌رود و نامۀ دخترش \noun{میلی} را می‌خواند و سپس از خانه خارج می‌شود.

    در ساعت 10 صبح، \noun{بلوم} نامه‌ای عاشقانه از ادارۀ پست دریافت می‌کند – او با زنی به نام \noun{مارتا کلیفورد} تحت نام مستعار هنری \noun{فلاور} نامه‌نگاری می‌کند. او نامه را می‌خواند، مختصری در کلیسایی می‌ماند و سپس لوسیونِ \noun{مالی} را به داروخانه‌چی سفارش می‌دهد. او با \noun{بانتام لاینز} برخورد می‌کند که اشتباهاً فکر می‌کند \noun{بلوم} دارد به او راجع به اسب \noun{ثرواوی} در مسابقۀ بعدازظهر \noun{گلدکاپ} راهنمایی می‌کند.

    حوالی ساعت 11 صبح، \noun{بلوم} همراه با \noun{سایمون ددالوس} (پدر \noun{استیون})، \noun{مارتین کانینگهام} و \noun{جک پاور} به مراسم خاکسپاری \noun{پدی دیگنام} می‌رود. مردها با \noun{بلوم} مثل یک غریبه برخورد می‌کنند. در مراسم خاکسپاری، \noun{بلوم} به مرگ پسرش و پدرش فکر می‌کند.

    ظهر، \noun{بلوم} را در دفتر روزنامۀ \noun{فریمن} می‌بینیم که دارد دربارۀ یک آگهی برای \noun{کیزِ} مشروب‌فروش مذاکره می‌کند. چندین مردِ علاف منجمله \noun{مایلس کرافوردِ} ویراستار، در دفتر می‌چرخند و بحث‌های سیاسی می‌کنند. \noun{بلوم} برای حتمی‌کردن آگهی خارج می‌شود. \noun{استیون} با نامۀ \noun{دیزی} وارد دفتر روزنامه می‌شود. \noun{استیون} و بقیۀ مردها همزمان با بازگشت \noun{بلوم} دارند به سمت میخانه می‌روند. مذاکرات آگهی \noun{بلوم} توسط \noun{کرافورد} که دارد بیرون می‌رود، رد می‌شود.

    در ساعت 1 بعدازظهر، \noun{بلوم} با \noun{جوسی برین}، عشق قدیمی‌اش برخورد می‌کند و راجع به \noun{مینا پیورفوی} که در زایشگاه بستری است، صحبت می‌کنند. \noun{بلوم} وارد رستوران \noun{برتون} می‌شود ولی تصمیم می‌گیرد به سمت \noun{دیوی برن} برود تا نهاری سبک بخورد. \noun{بلوم} به یاد بعدازظهری عاشقانه با \noun{مالی} در \noun{هاوث} می‌افتد. \noun{بلوم} خارج شده و دارد به سمت کتابخانۀ ملی می‌رود که \noun{بویلان} را در خیابان می‌بیند و به داخل موزۀ ملی پناه می‌برد.

    در ساعت 2 بعدازظهر، \noun{استیون} دارد «تئوری هملت» خود را به طور غیررسمی برای \noun{اِی.ای} شاعر و \noun{جان اگلینتون}، \noun{بست} و \noun{لیستر} کتابدار توضیح می‌دهد. \noun{اِی.ای}، تئوری \noun{استیون} را سبک می‌شمارد و خارج می‌شود. \noun{باک} وارد می‌شود و با تمسخر، \noun{استیون} را به خاطر قال گذاشتن او و \noun{هینز} در میخانه سرزنش می‌کند. در راه خروجی، \noun{باک} و \noun{استیون} از کنار \noun{بلوم} می‌گذرند که آمده تا رونوشتی از آگهی \noun{کیز} بردارد.

    در ساعت 4 بعدازظهر، \noun{سایمون ددالوس}، \noun{بن دالرد}، \noun{لنه‌هان} و \noun{بلیزس بویلان} در نوشگاه هتل \noun{اورموند} گرد هم می‌آیند. \noun{بلوم} متوجه ماشین \noun{بویلان} در بیرون هتل می‌شود و تصمیم می‌گیرد او را زیر نظر بگیرد. \noun{بویلان} خیلی زود برای قرارش با \noun{مالی} خارج می‌شود و \noun{بلوم} با عصبانیت در رستوران \noun{اورموند} می‌نشیند – او موقتاً با آوازخوانی \noun{ددالوس} و \noun{دالرد} آرام می‌شود. \noun{بلوم} جواب نامۀ \noun{مارتا} را می‌نویسد و می‌رود که نامه را پست کند.

    در ساعت 5 بعدازظهر، \noun{بلوم} به میخانۀ \noun{بارنی کیرنان} می‌رود تا با \noun{مارتین کانینگهام} درباره مسائل \noun{مالی} خانوادۀ \noun{دیگنام} صحبت کند، ولی \noun{کانینگهام} هنوز نرسیده است. \noun{شهروند}، یک میهن‌پرست خشونت‌گرای ایرلندی، سیاه مست می‌شود و به یهودی بودنِ \noun{بلوم} می‌تازد. \noun{بلوم} جلوی \noun{شهروند} می‌ایستد و از صلح و عشق دربرابر خشونت بیگانه‌هراسانه دفاع می‌کند. \noun{بلوم} و \noun{شهروند}، قبل از اینکه کالسکۀ \noun{کانینگهام}، \noun{بلوم} را از صحنه دور کند، با هم در خیابان مشاجره می‌کنند.

    حوالی غروب آفتاب، \noun{بلوم} بعد از رفتن به خانۀ خانم \noun{دیگنام}، در ساحل \noun{سندی‌مونت} استراحت می‌کند. زنی جوان که \noun{گرتی مک‌داول} نام دارد متوجه می‌شود که \noun{بلوم} دارد از ساحل به او نگاه می‌کند. \noun{گرتی} عمداً پایش را بیشتر و بیشتر نشان \noun{بلوم} می‌دهد در حالیکه \noun{بلوم} دارد یواشکی استمنا می‌کند. \noun{گرتی} می‌رود و \noun{بلوم} چرت می‌زند.

    ساعت 10 شب، \noun{بلوم} به زایشگاه می‌رود تا به \noun{مینا پیورفوی} سر بزند. \noun{استیون} و چند نفر از دوستانش که دانشجوی پزشکی‌اند نیز در بیمارستان هستند و مشغول نوشیدن و وراجی با صدای بلند راجع به مسائل مرتبط با تولد هستند. \noun{بلوم} قبول می‌کند که به آنها ملحق شود، هر چند که درنهان به خاطر تقلای خانم \noun{پیورفوی} در طبقۀ بالا، با عیاشی آنها مخالف است. \noun{باک} از راه می‌رسد و مردها به میخانۀ \noun{بورک} می‌روند. موقع تعطیل شدن میخانه، \noun{استیون}، دوستش \noun{لینچ} را راضی می‌کند که به فاحشه‌خانه بروند و \noun{بلوم} آنها را تعقیب می‌کند تا مراقبشان باشد.
    \noun{بلوم} بالاخره، \noun{استیون} و \noun{لینچ} را در فاحشه‌خانۀ \noun{بلا کوهن} می‌یابد. \noun{استیون} مست است و فکر می‌کند که دارد روح مادرش را می‌بیند – مملو از خشم و دیوانگی، چراغی را با چوبدستی‌اش خرد می‌کند. \noun{بلوم} دنبال \noun{استیون} می‌رود و او را در بحث با یک سرباز انگلیسی که \noun{استیون} را کتک می‌زند، می‌یابد.

    \noun{بلوم}، \noun{استیون} را به هوش می‌آورد و او را به استراحتگاه رانندگان تاکسی می‌برد تا قهوه‌ای بخورد و سر حال بیاید. \noun{بلوم}، \noun{استیون} را به خانه‌اش دعوت می‌کند.
    بعد از نیمه‌شب، \noun{استیون} و \noun{بلوم} به خانۀ \noun{بلوم} می‌روند. آنها کاکائوی داغ می‌خورند و دربارۀ گذشته‌شان صحبت می‌کنند. \noun{بلوم} از \noun{استیون} می‌خواهد که شب را بماند. \noun{استیون} مودبانه تقاضای او را رد می‌کند. \noun{بلوم} او را بدرقه می‌کند و به داخل برمی‌گردد تا شواهد حضور \noun{بویلان} را پیدا کند. \noun{بلوم} هنوز حالش خوب است و به رختخواب می‌رود، و داستان روزش را برای \noun{مالی} تعریف کرده و از او می‌خواهد صبحانه‌اش را به رختخواب بیاورد.

    بعد از اینکه \noun{بلوم} خوابش می‌برد، \noun{مالی} بیدار می‌ماند و از تقاضای \noun{بلوم} برای آوردن صبحانه به رختخواب در تعجب است. ذهن او به دوران کودکی‌اش در \noun{جیبرالتر}، سکس بعدازظهرش با \noun{بویلان}، حرفۀ خوانندگی‌اش و \noun{استیون ددالوس} مشغول است. تفکراتش راجع به \noun{بلوم} در طی مونولوگی که با خود دارد به تندی تغییر می‌کند ولی در انتها با یادآوری لحظات عاشقانه‌ای که در \noun{هاوث} داشتند و با دیدی مثبت به پایان می‌رسد.
\end{document}