\documentclass[12pt,onecolumn,a4paper]{article}
\usepackage{epsfig,graphicx,subfigure,amsthm,amsmath}
\usepackage{color,xcolor}
\usepackage{xepersian}
\settextfont[Scale=1.2]{IRLotus}
\setlatintextfont[Scale=1]{Arial}

\begin{document}
    \title{پارس مورف: تحلیلگر صرفی زبان فارسی}
    \author{
    وحید مواجی\\
    محرم اسلامی\\
    بهرام وزیرنژاد
    }
    \date{27 اردیبهشت 1391}
    \maketitle

    \section{چکیده}
    در مقالهٔ حاضر می‌کوشیم مبنای نظری، نحوهٔ طراحی و عملکرد سامانهٔ تحلیل‌گر صرفی زبان فارسی با عنوان اختصاری "پارس مورف" را معرفی کنیم. پارس مورف سامانه‌ای مبتنی بر قواعد صرفی زبان فارسی است که ساخت درونی کلمات فارسی را با توجه به نظام تصریف و نظام واژه‌سازی زبان تجزیه و تحلیل می‌کند و مقولهٔ دستوری و نقش هر کدام از اجزای سازندهٔ کلمه را مشخص می‌کند. پارس مورف با استفاده از یک واژگان حدوداً 45000 واژه‌‌ای و نیز در چارچوب قواعد صرفی زبان فارسی که بر یک تحقیق جامع زبان‌شناختی استوار است می‌تواند واژه‌های پیچیده و نیز صورت‌های ممکن تصریفی و حتی واژه‌های خارج از واژگان را تحلیل کند.
    دقت نسخهٔ اول پارس مورف حدود 95\% است که افزودن اطلاعات نحوی و مسائل مربوط به هم‌نویسه‌ها و نیز لحاظ کردن ویژگی‌های خط فارسی می‌تواند این دقت را به 100\% نزدیک کند. از پارس مورف می‌توان در مطالعات محض زبانی و نیز پردازش ماشینی زبان فارسی استفاده کرد.

\end{document}