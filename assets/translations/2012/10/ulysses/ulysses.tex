% Default Compiler: txs:///xelatex
% Default Bibliography Tool: BibTex

\documentclass[12pt,onecolumn,a4paper]{book}
\usepackage{epsfig,graphicx,subfigure,amsthm,amsmath}
\usepackage{caption}
\usepackage{color,xcolor}
\usepackage[linktocpage=true,colorlinks,citecolor=blue,pagebackref=true]{hyperref}
\usepackage[top=30mm, bottom=30mm, left=30mm, right=30mm]{geometry}
\usepackage[T1]{fontenc}
\usepackage[utf8]{inputenc}
\newcommand{\noun}[1]{\textit{\textcolor{gray}{#1}}}

\usepackage{authblk}
\usepackage{polyglossia}
\setotherlanguage{english}
\setotherlanguage{persian}
\usepackage[nonamebreak]{natbib}
\usepackage{xepersian}
\settextfont[Scale=1.2]{IRLotus}
\defpersianfont\mfo[Scale=1.2]{IRLotus}
\setlatintextfont[Scale=1]{Doulos SIL}

\begin{document}
    \title{خلاصۀ اولیس \LTRfootnote{\lr{SparkNotes Editors. ``SparkNote on Ulysses.'' SparkNotes.com. SparkNotes LLC. 2003.}}}
    \author{جیمز جویس\\
    ترجمه وحید مواجی
    }
    \date{مهر 1391}
    \frontmatter                            % only in book class (roman page #s)
    \maketitle                              % Print title page.
    \tableofcontents                        % Print table of contents
    \mainmatter


    \part{}
    \chapter{شخصیت‌ها}
    \noun{لئوپلد بلوم}\LTRfootnote{\lr{Leopold Bloom}} - مردی سی و هشت ساله و مسئول تبلیغات در دوبلین. \noun{بلوم} در دوبلین با \noun{رودلف}\LTRfootnote{\lr{Rudolph}}، پدر یهودیِ مجارستانی‌اش‌ و \noun{الن}\LTRfootnote{\lr{Ellen}}، مادر کاتولیکِ ایرلندی‌اش بزرگ شده است. او از مطالعه و تفکر دربارۀ علوم و اختراعات و شرح معلوماتش به دیگران لذت می‌برد. \noun{بلوم}، عاطفی و کنجکاو است و عاشق موسیقی می‌باشد. او ذهنش درگیر روابط سردش با زنش \noun{مالی} است.

    \part{}
    \chapter[تلماخوس]{تلماخوس\protect\footnote{\lr{Telemachus}-\rl{ در اسطوره‌های یونان پسر اولیس و پنلوپه است. در کودکی پسری ترسو بود. اما آتنه به او شجاعت بخشید. در دوران سرگردانی پدر به جست‌وجویش رفت.}}}

    \chapter[نستور]{نستور\protect\footnote{\lr{Nestor}-\rl{در اسطوره‌های یونان، شاه پولوس است. پسر نرئوس و خلوریس بود. از میان دوازده پسر نرئوس تنها او بود که از حملۀ هراکلس به پولوس جان سالم در برد. در جنگ تروا سالخورده‌ترین و عاقل‌ترین جنگاور محسوب می‌شد.}}}

    \chapter[پروتئوس]{پروتئوس\protect\footnote{\lr{Proteus}-\rl{در اسطوره‌های یونان، یکی از خدایان دریا که اشکال مختلف به خود می‌گرفته است. زمانی او را پسر پوزئیدون دانسته‌اند و زمانی ملازم و همنشین وی. او هم از قدرت پیشگویی برخوردار بود و هم از قدرت تغییر شكل در هر زمان كه اراده می كرد.}}}

    \chapter[کالوپسو]{کالوپسو\protect\footnote{\lr{Calypso}-\rl{در اسطوره‌های یونان، یک پری دریایی است. دختر اطلس بود و در جزیره اوگوگیا زندگی می‌کرد. اولیس در راه بازگشت از تروا، به این جزیره وارد شد. کالوپسو به او دل بست و او را هفت سال نزد خود نگاه داشت. به اولیس پیشنهاد کرد همیشه با او بماند و جاودان شود. اما اولیس در هوای خانه بود. عاقبت زئوس، هرمس را فرستاد تا کالوپسو را راضی کند دست از اولیس بردارد.}}}

    \chapter[لوتوفاگ‌ها]{لوتوفاگ‌ها\protect\footnote{\lr{Lotus-eaters} یا \lr{Lotophagi} یا \lr{Lotophaguses}-\rl{به معنی خورندگان نیلوفر آبی، در اسطوره‌های یونان، نام قبیله‌ای است که در ساحل لیبی زندگی می‌کردند. اولیس با همراهانش به جزیره لوتوفاگ‌ها وارد شد. آن‌ها از میوه نیلوفر آبی خوردند و حافظه خود را از دست دادند.}}}

    \chapter[هادس]{هادس\protect\footnote{\lr{Hades}-\rl{ در اساطیر یونانی، فرمانروای مردگان و دنیای زیرزمین، فرزند کرونوس و رئا است. او در قرعه‌کشی با برادرانش، بدترین سهم را برنده شد و آن جهان زیرین یا دنیای مردگان بود درصورتی که برادران او زئوس و پوزئیدون به ترتیب آسمان و دریا نصیبشان شد. از آنجایی که رعایای هادس را مردگان تشکیل می‌دادند، او به کسانی که موجب افزایش جمعیت سرزمینش می‌شدند بسیار علاقه داشت. مانند ارینی‌ها Erinnyes یا خشم و ناامیدی، که کارشان تعقیب گناهکاران و سوق دادن آن‌ها به سمت خودکشی بود.}}}

    \chapter[آیولوس]{آیولوس\protect\footnote{\lr{Aeolus}-\rl{ در اسطوره‌های یونان، پادشاه جزیره شناور آیولیا و خدای زمینی بادها است. زئوس قدرت مهار بادها را به او داده بود و خدایی زمینی محسوب می‌شد. با مهار باد به اولیس در رسیدن به تروا کمک کرد.}}}

    \chapter[لاستریگون‌ها]{لاستریگون‌ها\protect\footnote{\lr{Laestrygonians} یا \lr{Laestrygones} یا \lr{Laistrygones}-\rl{ در اساطیر یونان، قبیله‌ای از غول‌های آدمخوار. اولیس در راه بازگشت به ایتاکا با آنها مواجه شد. غول‌ها، بسیاری از مردان اولیس را خوردند و یازده کشتی از دوازه کشتی او را با پرتاب سنگ از بالای تپه نابود کردند.}}}

    \chapter[سیلا و کاریبد]{سیلا و کاریبد\protect\footnote{\lr{Scylla and Charybdis}-\rl{یا اسکیلا و کاریبدس یا سیلا و شاریبدیس، دو هیولای دریایی از اساطیر یونان هستند که توسط هومر مورد اشاره قرار گرفته اند. بعد از سنت‌های یونانی، محل آنها را در دو طرف تنگه مسینا و مقابل یکدیگر، در نظر گرفته‌اند. این محل بین سیسیل و سرزمین اصلی گراسیا مگنا یا همان یونان بزرگ (در جنوب ایتالیا) واقع است. گفته می‌شود که سیلا و کاریبد، در واقع در کنار یکدیگر، تهدیدی جدی و غیر قابل اجتناب در مسیر عبور ملوانان به حساب می‌آمدند؛ بدین ترتیب، کاریبد و سیلا هر دو در خود معنای جلوگیری کننده از عبور را دارند.}}}

    \chapter[صخره‌های سرگردان]{صخره‌های سرگردان\protect\footnote{\lr{Wandering Rocks} یا \lr{Planctae}-\rl{در اساطیر یونان، گروهی از صخره‌ها بودند که دریای بین آنها بیرحمانه خروشان بود.}}}

    \chapter[سیرن‌ها]{سیرن‌ها\protect\footnote{\lr{Sirens}-\rl{یا سایرن، یا حوری دریایی اساطیر یونان، گاهی به صورت موجودی با بدن یک پرنده و سر یک زن، و در سایر موارد به شکل تنها یک زن تصویر شده‌است. سیرن‌ها دختران خدای دریا فورکیس بوده‌اند، هرچند در نسخۀ دیگری از اساطیر، پدرشان خدای نهر، آکلوس دانسته شده است. آنها آوازی بسیار زیبا و فریبنده داشتند و دریانوردان را با آوای خود گمراه کرده و به کام صخره‌های مرگ‌آوری که بر روی آن آواز میخواندند، میکشیدند. اولیس، قهرمان افسانه‌ای یونان، توانست بدون هیچ خطری از جزیره آنان بگذرد، از آنرو که طبق نصیحت سیرسه ساحره، او از همراهانش خواست تا گوشهایشان را با موم پرکرده و او را محکم به دکل کشتی ببندند تا با اغوای آنان کشتی را به بیراهه نکشاند و بی هیچ خطری بتواند آواز آنان را بشنود.}}}

    \chapter[سیکلوپ]{سیکلوپ\protect\footnote{\lr{Cyclops}-\rl{یکی از موجودات افسانه‌ای در اساطیر یونانی است. سیکلوپ‌ها در اسطوره‌های یونان، غول‌هایی با یک چشم در وسط پیشانی هستند. آنها قدرتمند و سرسخت بودند. حرکات آنها همیشه همراه با خشونت و قدرت بود.}}}

    \chapter[ناوسیکائا]{ناوسیکائا\protect\footnote{\lr{Nausicaa}-\rl{در اسطوره‌های یونان، دختر آلکینوئوس است. زمانی که اولیس به جزیرۀ سخریا رسید او عاشق اولیس شد و از پدر خواست اجازه دهد با وی ازدواج کند. اما اولیس که قصد بازگشت به سرزمین خود را داشت نپذیرفت.}}}

    \chapter[گلۀ گاوِ خورشید]{گلۀ گاوِ خورشید\protect\footnote{\lr{Oxen of the Sun} یا \lr{Cattle of Helios}-\rl{در اساطیر یونان، در جزیرۀ تریناکیا می‌چریدند که معادل سیسیل امروزین است. هلیوس یا خدای خورشید، هفت گله گاو و هفت رمه گوسفند داشت.}}}

    \chapter[کیرکه]{کیرکه\protect\footnote{\lr{Circe}-\rl{در اسطوره‌های یونان، دختر هلیوس و پرسه است. جادوگری قدرتمند بود که دشمنانش را به حیوان تبدیل می‌کرد. پیکوس، سکولا و همراهان اولیس از جمله کسانی بودند که مورد خشم او قرار گرفتند.}}}

    \chapter[ائومایوس]{ائومایوس\protect\footnote{\lr{Eumaeus}-\rl{خوک‌چران اولیس. خدمتکار اولیس بود و او را در راه رسیدن به همسرش پنلوپه یاری داد. در اصل شاهزاده بود ولی کنیزی او را ربود و به لائرتس، پدر اولیس فروخت.}}}

    \chapter[ایتاکا]{ایتاکا\protect\footnote{\lr{Ithaca}-\rl{شهری که اولیس در آن زندگی می‌کرد.}}}

    \chapter[پنلوپه]{پنلوپه\protect\footnote{\lr{Penelope}-\rl{در اسطوره‌های یونان، همسر اولیس و مادر تلماخوس است. او در جوانی به دلیل زیبایی‌اش خواستگاران زیادی داشت. پدرش برای جلوگیری از دعوا و کشمکش مسابقه‌ای ترتیب داد تا برندۀ آن را به دامادی انتخاب کند. در این مسابقه اولیس سرافراز بیرون آمد و با پنلوپه ازدواج کرد.}}}

    \part{}
    \chapter{طرح کلی داستان}
\end{document}