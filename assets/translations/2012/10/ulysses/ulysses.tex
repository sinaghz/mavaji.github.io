% Default Compiler: txs:///xelatex
% Default Bibliography Tool: BibTex

\documentclass[12pt,onecolumn,a4paper]{book}
\usepackage{epsfig,graphicx,subfigure,amsthm,amsmath}
\usepackage{caption}
\usepackage{color,xcolor}
\usepackage[linktocpage=true,colorlinks,citecolor=blue,pagebackref=true]{hyperref}
\usepackage[top=30mm, bottom=30mm, left=30mm, right=30mm]{geometry}
\usepackage[T1]{fontenc}
\usepackage[utf8]{inputenc}
\newcommand{\noun}[1]{\textit{\textcolor{gray}{#1}}}

\usepackage{authblk}
\usepackage{polyglossia}
\setotherlanguage{english}
\setotherlanguage{persian}
\usepackage[nonamebreak]{natbib}
\usepackage{xepersian}
\settextfont[Scale=1.2]{IRLotus}
\defpersianfont\mfo[Scale=1.2]{IRLotus}
\setlatintextfont[Scale=1]{Doulos SIL}

\begin{document}
    \title{خلاصۀ اولیس \LTRfootnote{\lr{SparkNotes Editors. ``SparkNote on Ulysses.'' SparkNotes.com. SparkNotes LLC. 2003.}}}
    \author{جیمز جویس\\
    ترجمه وحید مواجی
    }
    \date{مهر 1391}
    \frontmatter                            % only in book class (roman page #s)
    \maketitle                              % Print title page.
    \tableofcontents                        % Print table of contents
    \mainmatter


    \part{}
    \chapter{شخصیت‌ها}

    \part{}
    \chapter[تلماخوس]{تلماخوس\protect\footnote{\lr{Telemachus}-\rl{ در اسطوره‌های یونان پسر اولیس و پنلوپه است. در کودکی پسری ترسو بود. اما آتنه به او شجاعت بخشید. در دوران سرگردانی پدر به جست‌وجویش رفت.}}}

    \chapter{نستور}
    \chapter{پروتئوس}
    \chapter{کالوپسو}
    \chapter{لوتوفاگ‌ها}
    \chapter{هادس}
    \chapter{آیولوس}
    \chapter{لاستریگون‌ها}
    \chapter{سیلا و کاریبد}
    \chapter{صخره‌های سرگردان}
    \chapter{سیرن‌ها}
    \chapter{سیکلوپ}
    \chapter{ناوسیکائا}
    \chapter{گلۀ گاوِ خورشید}
    \chapter{کیرکه}
    \chapter{ائومایوس}
    \chapter{ایتاکا}
    \chapter{پنلوپه}

    \part{}
    \chapter{طرح کلی داستان}
\end{document}