% Default Compiler: txs:///xelatex
% Default Bibliography Tool: BibTex

\documentclass[12pt]{book}
\usepackage[x11names]{xcolor}
\usepackage[linktocpage=true,colorlinks,citecolor=blue,pagebackref=true]{hyperref}
\usepackage[top=30mm, bottom=30mm, left=30mm, right=30mm]{geometry}
\usepackage[utf8]{inputenc}
\newcommand{\noun}[1]{\textit{\textcolor{black!70}{#1}}}

\usepackage{xepersian}
\settextfont[Scale=1.5]{Far.Mashin Tahrir}
\defpersianfont\mfo[Scale=1.5]{Far.Mashin Tahrir}
\setlatintextfont[Scale=1]{Courier New}

\begin{document}
    \title{خلاصۀ اولیس \LTRfootnote{\lr{SparkNotes Editors. ``SparkNote on Ulysses.'' SparkNotes.com. SparkNotes LLC. 2003.}}}
    \author{جیمز جویس\\
    ترجمه وحید مواجی
    }
    \date{مهر 1391}
    \frontmatter                            % only in book class (roman page #s)
    \maketitle                              % Print title page.
    \tableofcontents                        % Print table of contents
    \mainmatter


    \part{}
    \chapter{شخصیت‌ها}
    \paragraph{\noun{لئوپلد بلوم}\protect\LTRfootnote{\lr{Leopold Bloom}}}
    مردی سی و هشت ساله و مسئول تبلیغات در دوبلین. \noun{بلوم} در دوبلین با \noun{رودلف}\LTRfootnote{\lr{Rudolph}}، پدر یهودیِ مجارستانی‌اش‌ و \noun{الن}\LTRfootnote{\lr{Ellen}}، مادر کاتولیکِ ایرلندی‌اش بزرگ شده است. او از مطالعه و تفکر دربارۀ علوم و اختراعات و شرح معلوماتش به دیگران لذت می‌برد. \noun{بلوم}، عاطفی و کنجکاو است و عاشق موسیقی می‌باشد. او ذهنش درگیر روابط سردش با زنش \noun{مالی} است.
    \paragraph{\noun{مریان (مالی) بلوم}\protect\LTRfootnote{\lr{Marion (Molly) Bloom}}}
    همسر لئوپلد بلوم. مالی بلوم سی‌ساله است، کمی تپل و سبزه، خوش بر و رو و اهل لاس‌زدن می‌باشد. او تحصیلات زیادی ندارد ولی به هر تقدیر باهوش و صاحب‌نظر است. او خواننده‌ای حرفه‌ای است که توسط پدر ایرلندی‌اش، سرگرد برایان توییدی\LTRfootnote{\lr{Brian Tweedy}} در جیبرالتر\LTRfootnote{\lr{Gibraltar}} بزرگ شده است. مالی حوصلۀ بلوم را ندارد مخصوصاً به این دلیل که از مرگ یازده سال پیشِ پسرشان رودی\LTRfootnote{\lr{Rudy}} به این طرف، بلوم دیگر با او صمیمی (در رابطۀ جنسی) نیست.
    \paragraph{\noun{استیوِن ددالوس}\protect\LTRfootnote{\lr{Stephen Dedalus}}}
    شاعری پرالهام و بیست و چندساله. استیون باهوش و فوق‌العاده کتابخوان و علاقه‌مند به موسیقی است. به نظر می‌رسد که بیشتر در دنیای ذهنی خودش زندگی می‌کند تا اینکه عضو انجمنی یا حتی گروه دانشجویان پزشکی که همقطارانش هستند باشد. استیون در کودکی، بسیار مذهبی بوده است ولی بر اثر مرگ مادرش که کمتر از یک سال پیش رخ داده، اکنون با مسائل مربوط به شک و ایمان دست و پنجه نرم می‌کند.
    \paragraph{\noun{مالاکای (باک)  مالیگان}\protect\LTRfootnote{\lr{Malachi (Buck) Mulligan}}}
    دانشجوی پزشکی و دوست استیون. باک مالیگان کمی چاق و اهل مطالعه است و تقریباً همه چیز را دست می‌اندازد. او به خاطر لطیفه‌های بی‌ادبی و بامزه‌ای که تعریف می‌کند تقریباً مورد علاقه همه به جز استیون، سایمون و بلوم است.
    \paragraph{\noun{هینز}\protect\LTRfootnote{\lr{Haines}}}
    دانشجوی فرهنگ فولکلور که به خصوص علاقه‌مند به مطالعۀ قوم و فرهنگ ایرلندی است. هینز اغلب اوقات ناخواسته مغرور و خودبین است. او در قلعه مارتلو\LTRfootnote{\lr{Martello}} اقامت دارد جایی که استیون و باک هم در آنجا زندگی می‌کنند
    \paragraph{\noun{هیو (بلیزس) بویلان}\protect\LTRfootnote{\lr{Hugh (``Blazes'') Boylan}}}
    مدیر کنسرت قریب‌الوقع مالی در بلفاست. بلیزس بویلان در شهر مشهور و محبوب است علی‌رغم اینکه کمی هرزه به نظر می‌رسد، مخصوصاً نسبت به زنان. بویلان به مالی علاقه‌مند شده است و آنها در بعدازظهرِ داستان رابطه‌ای با هم برقرار می‌کنند.
    \paragraph{\noun{میلیسنت (میلی) بلوم}\protect\LTRfootnote{\lr{Millicent (Milly) Bloom}}}
    دختر پانزده‌سالۀ مالی و لئوپلد بلوم که فی‌الواقع در اولیس ظاهر نمی‌شود. خانوادۀ بلوم اخیراً میلی را برای زندگی و یادگیری عکاسی به مالینگار\LTRfootnote{\lr{Mullingar}} فرستاده‌اند. میلی، بلوند و زیبا و علاقه‌مند به پسرها است - او با الک بانون\LTRfootnote{\lr{Alec Bannon}} در مالینگار قرار و مدار می‌گذارد.
    \paragraph{\noun{سایمون ددالوس}\protect\LTRfootnote{\lr{Simon Dedalus}}}
    پدرِ استیون ددالوس. سایمون ددالوس در کورک\LTRfootnote{\lr{Cork}} بزرگ شده و بعداً به دوبلین آمده و تا کنون مرد نسبتاً موفقی بوده است. مردان دیگر، او را سرلوحه خود قرار می‌دهند، هرچند که بعد از مرگ زنش، خانه و زندگی‌اش بی‌نظم و نامرتب شده است. سایمون دارای صدایی خوب و استعداد لطیفه‌گویی است و اگر عادت مشروب‌خوری نداشت می‌توانست از این همه استعداد سود ببرد. سایمون به شدت منتقد استیون است.
    \paragraph{\noun{اِی.ای (جرج راسل)}\protect\LTRfootnote{\lr{A.E. (George Russell)}}}
    اِی.ای نام مستعار جرج راسل، شاعر معروفِ احیای ادبیات ایرلندی است که در کانونِ حلقه‌های ادبی ایرلند می‌باشد - حلقه‌های ادبی که استیون را به خود راه نمی‌دهند. او عمیقاً به عرفان اسرارآمیز علاقه‌مند است. بقیه مردها چنان با او مشورت می‌کنند که انگار حرفش وحی منزل است.
    \paragraph{\noun{ریچارد بِست}\protect\LTRfootnote{\lr{Richard Best}}}
    کتابداری در کتابخانۀ ملی. بست، شخص مشتاق و علاقه‌مندی است، با این حال بخش عمده‌ای از مشارکتش در بحثِ هملت در فصل~\ref{ep:9}، نشانه‌هایی از باورهای غلطی دارد که به خیال خودش درست می‌باشند.
    \paragraph{\noun{ادی بوردمن}\protect\LTRfootnote{\lr{Edy Boardman}}}
    یکی از دوستان گرتی مک‌داول\LTRfootnote{\lr{Gerty MacDowell}}. رفتار مغرورانۀ گرتی، ادی را که می‌خواهد او را با گوشه و کنایه ضایع کند، می‌رنجاند.
    \paragraph{\noun{جوسی (نام خانۀ پدری: پاول) و دنیس برین}\protect\LTRfootnote{\lr{Josie (née Powell) and Denis Breen}}}
    جوسی پاول و بلوم وقتی جوان‌تر بودند به هم علاقه داشتند. جوسی زیبا و اهل لاس‌زدن بود. بعد از اینکه بلوم با مالی ازدواج کرد، جوسی هم با دنیس ازدواج کرد. دنیس برین کمی دیوانه است و پارانوید به نظر می‌رسد. مراقبت از چنین شوهر ابلهی اثر خود را روی جوسی گذاشته است و اکنون نحیف و خسته به نظر می‌رسد.
    \paragraph{\noun{سیسی، جکی و تامی کافری}\protect\LTRfootnote{\lr{Cissy, Jacky, and Tommy Caffrey}}}
    سیسی کافری یکی از بهترین دوستان گرتی مک‌داول است. او دختری با رفتار پسرانه و کمی رُک است. او مراقب برادران نوپای کوچکش، جکی و تامی است.
    \paragraph{\noun{شهروند}\protect\LTRfootnote{\lr{The citizen}}}
    یک میهن‌پرست ایرلندی مسن که از نهضت ناسیونالیست دفاع می‌کند. با اینکه به نظر نمی‌رسد شهروند هیچ ارتباط رسمی با نهضت داشته باشد ولی بقیه افراد، اخبار و اطلاعات را از او می‌پرسند. او سابقاً یکی از ورزشکاران و پهلوانان ایرلند بوده است. او ماجراجو و بیگانه‌هراس است.
    \paragraph{\noun{مارتا کلیفورد}\protect\LTRfootnote{\lr{Martha Clifford}}}
    زنی که بلوم با او تحت نام مستعار هنری فلاور\LTRfootnote{\lr{Henry Flower}} مکاتبه می‌کند. نامه‌های مارتا پر از غلطهای نگارشی است و تمایلات جنسی‌اش، غیرخلاقانه و ملال‌آورند.
    \paragraph{\noun{بلا کوهن}\protect\LTRfootnote{\lr{Bella Cohen}}}
    زن فاحشه‌ای خلافکار. بلا کوهن گنده، سبزه و دارای رفتاری مردانه است. او تا حدی طالب احترام از جانب بقیه است و پسری در آکسفورد دارد که شهریه‌اش را یکی از مشتریانش می‌پردازد.
    \paragraph{\noun{مارتین کانینگهام}\protect\LTRfootnote{\lr{Martin Cunningham}}}
    یکی از اعضای اصلی حلقۀ دوستان بلوم. مارتین کانینگهام نسبت به دیگران مهربان و باشفقت است و در لحظات مختلفی از روز (کل داستان در یک روز اتفاق می‌افتد) از بلوم دفاع می‌کند با این حال با بلوم مثل یک بیگانه رفتار می‌کند. قیافه او، شکسپیر را تداعی می‌کند.
    \paragraph{\noun{گرت دیزی}\protect\LTRfootnote{\lr{Garrett Deasy}}}
    مدیر مدرسۀ پسرانه‌ای که استیون در آنجا تدریس می‌کند. دیزی، پروتستانی از شمال ایرلند و به دولت انگلیس پایبند است. دیزی نسبت به استیون با تکبر برخورد می‌کند و شنوندۀ خوبی نیست. نامه پر و پیمان او به ویراستار دربارۀ تب برفکیِ احشام، موضوع استهزاء مردان دوبلینی در طی روز است.
    \paragraph{\noun{دیلی، کیتی، بودی و مگی ددالوس}\protect\LTRfootnote{\lr{Dilly, Katey, Boody, and Maggy Dedalus}}}
    خواهران جوان‌تر استیون. آنها بعد از مرگ مادرشان سعی در رتق و فتق امور منزلِ ددالوس دارند. به نظر می‌رسد که دیلی علائق و آرزوهایی مثل یادگیری زبان فرانسه دارد
    \paragraph{\noun{پاتریک دیگنام، خانم دیگنام و پاتریک دیگنام جونیور}\protect\LTRfootnote{\lr{Patrick Dignam, Mrs. Dignam, and Patrick Dignam, Jr.}}}
    پاتریک دیگنام یکی از آشنایان بلوم بود که خیلی زود بر اثر شرابخواری درگذشت. مراسم خاکسپاری او امروز است و بلوم و بقیه جمع می‌شوند تا برای بیوۀ دیگنام و بچه‌هایش مقادیری پول جمع کنند چرا که پدی\LTRfootnote{\lr{Paddy}} همه بیمۀ عمرش را صرف پرداخت دیونش کرده بود و برای بچه‌هایش چیزی باقی نگذاشته است.
    \paragraph{\noun{بن دالرد}\protect\LTRfootnote{\lr{Ben Dollard}}}
    مردی که در دوبلین به خاطر صدای بمِ عالی‌اش شهره است. کسب و کار بن دالرد مدتی پیش از رونق افتاده است. آدم خوش‌طینتی به نظر می‌رسد ولی احتمالاً به خاطر عادت شرابخواری گذشته‌اش، عصبی و پریشان است.
    \paragraph{\noun{جان اگلینتون}\protect\LTRfootnote{\lr{John Eglinton}}}
    مقاله‌نویسی که وقتش را در کتابخانۀ ملی می‌گذراند. جان اگلینتون، اعتماد به نفس و غرور جوانیِ استیون را تحقیر می‌کند و نسبت به تئوری هملت استیون با دیده تردید می‌نگرد.
    \paragraph{\noun{ریچی، سارا (سالی) و والتر گولدینگ}\protect\LTRfootnote{\lr{Richie, Sara (Sally), and Walter Goulding}}}
    ریچی گولدینگ، دایی استیون ددالوس است؛ او برادرِ ماری، مادرِ استیون بوده است. ریچی کارمند دادگستری است که اخیراً به خاطر مشکل کمرش کمتر توانسته کار کند - مسأله‌ای که به خاطر آن، موضوعِ خندۀ سایمون ددالوس شده است. والتر، پسر ریچی و سارا، لوچ است و لکنت زبان دارد.
    \paragraph{\noun{زو هیگینز}\protect\LTRfootnote{\lr{Zoe Higgins}}}
    فاحشه‌ای در فاحشه‌خانۀ بلا کوهن. زو بی‌پروا و در زخم زبان زدن استاد است.
    \paragraph{\noun{جو هاینز}\protect\LTRfootnote{\lr{Joe Hynes}}}
    گزارشگری از روزنامۀ دوبلین که اغلب اوقات بی‌پول است - او از بلوم، سه پوند قرض گرفته است و تا کنون آن را پس نداده. هاینز، بلوم را درست نمی‌شناسد و در اپیزود دوازدهم به نظر می‌رسد که دوست خوبی برای شهروند است.
    \paragraph{\noun{کورنی کله‌هر}\protect\LTRfootnote{\lr{Corny Kelleher}}}
    مسئول کفن و دفن که روابط خوبی با پلیس دارد.
    \paragraph{\noun{مینا کندی و لیدیا دوس}\protect\LTRfootnote{\lr{Mina Kennedy and Lydia Douce}}}
    دختران پیشخدمت هتل اورموند\LTRfootnote{\lr{Ormond}}. مینا و لیدیا اهل لاس‌زنی هستند و با مردانی که به نوشگاه می‌آیند گرم می‌گیرند، با این حال در خلوت خود از جنس مخالف به بدی یاد می‌کنند. دوشیزه دوس که موهای برنز رنگی دارد، بی‌پرواتر از آن یکی به نظر می‌رسد و با بلیزس بویلان درگیری داشته است. دوشیزه کندی که موهایی طلایی دارد، خوددارتر است.
    \paragraph{\noun{ند لمبرت}\protect\LTRfootnote{\lr{Ned Lambert}}}
    یکی از دوستان سایمون ددالوس و بقیه مردان در دوبلین. ند لمبرت اغلب در حال لطیفه‌گویی و خنده است. او در انبار غله و حبوبات در مرکز شهر کار می‌کند، در جایی که زمانی صومعۀ مریم مقدس بوده است.
    \paragraph{\noun{لنه‌هان}\protect\LTRfootnote{\lr{Lenehan}}}
    ویراستار مسابقات در روزنامۀ دوبلین؛ با این حال اسب مورد نظر او، سپتر\LTRfootnote{\lr{Sceptre}} در مسابقات گلدکاپ می‌بازد. لنه‌هان آدم بذله‌گویی است و با زنان لاس می‌زند. او بلوم را مسخره می‌کند ولی به سایمون و استیون ددالوس احترام می‌گذارد.
    \paragraph{\noun{لینچ}\protect\LTRfootnote{\lr{Lynch}}}
    دانشجوی پزشکی و دوست قدیمی استیون (او در «چهرۀ هنرمند در جوانی» هم حضور دارد). لینچ به شنیدن نظریات پرمدعا و فوقِ زیباشناسانۀ استیون عادت دارد و با سرسختی و لجاجتِ استیون آشناست. او با کیتی ریکتس\LTRfootnote{\lr{Kitty Ricketts}} قرار می‎‌گذارد.
    \paragraph{\noun{تامس دابلیو لیستر}\protect\LTRfootnote{\lr{Thomas W. Lyster}}}
    کتابداری در کتابخانۀ ملی دوبلین و عضو فرقۀ کویکر\LTRfootnote{\lr{Quaker}}. لیستر بیشترین علاقه را به صحبت‌های استیون در اپیزود نهم نشان می‌دهد.
    \paragraph{\noun{گرتی مک‌داول}\protect\LTRfootnote{\lr{Gerty MacDowell}}}
    زنی در اوان بیست سالگی و از خانواده‌ای از طبقۀ متوسطِ رو به پایین. گرتی از لنگی دائمی پایش رنج می‌برد که احتمالاً بر اثر تصادف با دوچرخه بوده است. او با دقت بسیاری به لباس پوشیدن و رژیمش اهمیت می‌دهد و آرزوی عاشق شدن و ازدواج دارد. او به ندرت به خودش اجازه می‌دهد راجع به معلولیتش فکر کند.
    \paragraph{\noun{جان هنری منتون}\protect\LTRfootnote{\lr{John Henry Menton}}}
    مشاور حقوقی در دوبلین که توسط پدی دیگنام استخدام شده است. وقتی بلوم و مالی عاشق هم بودند، منتون تحت تأثیر علاقه به مالی، رقیبی عشقی برای بلوم بود. او نسبت به بلوم با بی‌احترامی رفتار می‌کند
    \paragraph{\noun{راوی بی‌نام اپیزود دوازدهم}}
    راویِ بی‌نام اپیزود دوازدهم، در حال حاضر کارگزار وصول طلب است و این جدیدترین شغلش از بین شغل‌های بسیاری است که داشته. او از اینکه «بااطلاع» به نظر برسد لذت می‌برد و بخش عمدۀ شایعاتی که دربارۀ خانواده بلوم می‌داند را از دوستش «پیسر» بورک\LTRfootnote{\lr{``Pisser'' Burke}} شنیده که آنها را وقتی در هتل سیتی آرمز\LTRfootnote{\lr{City Arms}} زندگی می‌کردند می‌شناخته است.
    \paragraph{\noun{عضو شورای شهر، نانتی}\protect\LTRfootnote{\lr{City Councillor Nannetti}}}
    مسئول ارشد چاپ در روزنامه دوبلین و عضو پارلمان. نانتی یک دورگۀ ایتالیایی-ایرلندی است.
    \paragraph{\noun{جی.جی اُمالوی}\protect\LTRfootnote{\lr{J. J. O’Molloy}}}
    وکیلی که اکنون بیکار و بی‌پول است. اُمالوی، امروز در قرض گرفتن پول از دوستانش ناکام است. او در اپیزود دوازدهم در میخانۀ بارنی کیرنان\LTRfootnote{\lr{Barney Kiernan}}، از بلوم دفاع می‌کند.
    \paragraph{\noun{جک پاور}\protect\LTRfootnote{\lr{Jack Power}}}
    یکی از دوستان سایمون ددالوس و مارتین کانینگهام و دیگر مردان شهر. پاور احتمالاً در اجرای احکام کار می‌کند. او زیاد با بلوم خوب نیست.
    \paragraph{\noun{کیتی ریکتس}\protect\LTRfootnote{\lr{Kitty Ricketts}}}
    یکی از فاحشه‌هایی که در فاحشه‌خانۀ بلا کوهن کار می‌کنند. به نظر می‌رسد که کیتی با لینچ رابطه دارد و بخشی از روز را با او گذرانده است. او لاغر است و طرز لباس پوشیدنش، تمایلاتش به طبقۀ بالای جامعه را نشان می‌دهد.
    \paragraph{\noun{فلوری تالبوت}\protect\LTRfootnote{\lr{Florry Talbot}}}
    یکی از فاحشه‌های فاحشه‌خانۀ بلا کوهن. فلوری چاق است و کودن به نظر می‌رسد ولی به راحتی خوشحال می‌شود.

    \part{}
    \chapter[تلماخوس]{تلماخوس\protect\footnote{\lr{Telemachus}-\rl{ در اسطوره‌های یونان پسر اولیس و پنلوپه است. در کودکی پسری ترسو بود. اما آتنه به او شجاعت بخشید. در دوران سرگردانی پدر به جست‌وجویش رفت.}}}\label{ep:1}
    ساعت حدود 8 صبح است و باک مالیگان، در حال تقلید و مسخرۀ مراسم عشاء ربانی با کاسۀ ریش‌تراشی‌اش، استیون ددالوس را صدا می‌زند تا بالای سقف قلعۀ مارتلو که مشرف بر خلیج دوبلین است، بیاید.  استیون به مسخره‌بازیِ پرخاشگرانۀ باک بی‌توجه است - او حوصلۀ هینز را ندارد، فرد انگلیسی‌ای که باک دعوتش کرده تا در قلعه بماند. استیون با ناله‌های هینز دربارۀ کابوسی که در آن یک پلنگ سیاه دیده بود، از خواب شبانه بیدار شده است.

    مالیگان و استیون به دریا نگاه می‌کنند که باک از آن به مادر کبیر یاد می‌کند. این کار، مالیگان را به یاد غضب عمه‌اش نسبت به استیون می‌اندازد چرا که استیون قبول نکرده بود کنار بستر مرگ مادر خودش دعا کند. استیون که هنوز لباس عزا به تن دارد به دریا می‌نگرد و به مرگ مادرش فکر می‌کند، در حالی که باک، استیون را به خاطر لباس‌‌های دست دوم و ظاهر کثیفش دست می‌اندازد. باک یک آینۀ شکسته را جلوی استیون می‌گیرد تا خودش را در آن ببیند. استیون همدردی باک را رد می‌کند و اظهار می‌کند که چنین «آینۀ شکسته‌ای از یک نوکر» نشانه‌ای از هنر ایرلندی است. باک بازویش را به نشانه همدردی دور استیون حلقه می‌کند و می‌گوید که آنها با هم می‌توانند ایرلند را به سطحی از فرهنگ، همتای یونان باستان برسانند. باک پیشنهاد می‌دهد در صورتی که هینز، دوباره استیون را برنجاند، او را تهدید کنند و استیون به یاد «تحقیر و آزار» یکی از همکلاسی‌هایشان به نام کلایو کمپتورپ\LTRfootnote{\lr{Clive Kempthorpe}} توسط باک می‌افتد.

    باک از استیون درباره خشم خاموشش می‌پرسد و استیون سرآخر غضبش نسبت به باک را تأیید می‌کند - ماه‌ها قبل، استیون شنیده بود که باک مادرش را «مثل سگ، مرده» خطاب کرده بود. باک سعی می‌کند که از خودش دفاع کند، سپس تسلیم می‌شود و استیون را ترغیب می‌کند که از خشمگین بودن نسبت به تفاخر و غرور خودش دست بردارد.

    باک وارد قلعه می‌شود و بدون این که بداند، آوازی را می‌خواند که استیون برای مادر در حال مرگش خوانده بود. استیون احساس می‌کند که توسط مادر مرده‌اش یا خاطره او تسخیر شده است. باک، استیون را به طبقه پایین برای صبحانه فرا می‌خواند. او استیون را ترغیب می‌کند تا از هینز، که تحت تأثیر طبع ایرلندی استیون قرار دارد، درخواست پول کند، ولی استیون قبول نمی‌کند. استیون به آشپزخانه می‌رود و به باک برای صبحانه کمک می‌کند. هینز اعلام می‌کند که زن شیرفروش دارد می‌آید. باک لطیفه‌ای می‌گوید درباره «مادر پیر، گروگن\LTRfootnote{\lr{Old mother Grogan}}» که چای درست می‌کند و آب (ادرار) درست می‌کند و هینز را تشویق می‌کند که از آن لطیفه در کتابی دربارۀ زندگی مردم ایرلند استفاده کند.

    زن شیرفروش وارد می‌شود، و استیون او را به شکل نمادی از ایرلند تصور می‌کند. استیون از این که زن شیرفروش به باک، دانشجوی پزشکی، بیشتر از او احترام می‌گذارد در نهان ناراحت است. هینز با او (زن) به ایرلندی صحبت می‌کند ولی او (زن) حرفش را نمی‌فهمد و فکر می‌کند که (هینز) دارد فرانسوی صحبت می‌کند. باک پول او را پرداخت می‌کند و زن می‌رود.

    هینز می‌گوید که تمایل دارد تا از گفته‌های استیون کتابی بنویسد، ولی استیون می‌پرسد آیا از آن پولی به دست می‌آید یا نه. هینز بیرون می‌رود و باک، استیون را به خاطر گستاخ بودن و از بین بردن فرصتشان برای گرفتن پول میگساری از هینز سرزنش می‌کند. باک لباس می‌پوشد و هر سه مرد به سمت آب می‌روند. در راه، استیون توضیح می‌دهد که قلعه را از وزیر جنگ اجاره کرده است. هینز از استیون دربارۀ تئوری هملت‌اش می‌پرسد ولی باک اصرار می‌کند آن را به بعد از میگساری به تأخیر بیاندازند. هینز توضیح می‌دهد که قلعۀ مارتلوی آنها، او را به یاد ال-سینور\LTRfootnote{\lr{El-sinore}} هملت می‌اندازد. باک، حرف هینز را قطع می‌کند تا پیش بیفتد، برقصد و «تصنیف عیسی بذله‌گو» را بخواند.

    هینز و استیون با هم راه می‌روند. همزمان با صحبت هینز، استیون حدس می‌زند که باک کلید قلعه را بخواهد - قلعه‌ای که استیون پول اجاره‌اش را می‌دهد. هینز از استیون راجع به عقاید مذهبی‌اش می‌پرسد. استیون توضیح می‌دهد که دو ارباب، انگلستان و کلیسای کاتولیک، بر سر راه تفکر آزادش ایستاده اند و ارباب سوم، ایرلند، از او، «کارهای عجیب و غریب» می‌خواهد. هینز، در حالی که سعی می‌کند درباره بردگی ایرلند نسبت به بریتانیا دوستانه رفتار کند، به نرمی می‌گوید «به نظر باید تاریخ را سرزنش کرد». هینز و استیون به خلیج خیره می‌شوند و استیون مردی را به خاطر می‌آورد که به تازگی غرق شده است.

    هینز و استیون به سمت آب می‌روند، جایی که باک دارد لباس‌هایش را در می‌آورد و دو نفر دیگر، شامل یکی از دوستان باک، دارند شنا می‌کنند. باک با دوستش راجع به دوست مشترکشان، بانون که در وستمیث\LTRfootnote{\lr{Westmeath}} است، صحبت می‌کند - بانون ظاهراً دوست‌دختری دارد (که بعداً می‌فهمیم میلی بلوم است). باک به آب می‌زند، در حالی که هینز سیگار می‌کشد. استیون اعلام می‌کند که دارد می‌رود و باک از او درخواست کلید قلعه و دو پنی برای یک پینت\footnote{\lr{Pint}-\rl{واحد حجم معادل يک هشتم گالن}} آبجو می‌کند. باک با استیون در ساعت 12:30 در میخانۀ کشتی\LTRfootnote{\lr{The Ship}} قرار می‌گذارد. استیون می‌رود و عهد می‌کند که امشب به قلعه باز نگردد چرا که باکِ «غاصب»، آن را به چنگ آورده است.

    \chapter[نستور]{نستور\protect\footnote{\lr{Nestor}-\rl{در اسطوره‌های یونان، شاه پولوس است. پسر نرئوس و خلوریس بود. از میان دوازده پسر نرئوس تنها او بود که از حملۀ هراکلس به پولوس جان سالم در برد. در جنگ تروا سالخورده‌ترین و عاقل‌ترین جنگاور محسوب می‌شد.}}}\label{ep:2}
    استیون در حال تدریس در کلاس تاریخ درباره پیروزی پیروس\footnote{\lr{Pyrrhus}-\rl{ژنرال و سیاستمدار یونانی}} است - کلاس خیلی نظم و ترتیب ندارد. او به دانش‌آموزان تمرین می‌دهد و پسری به نام آرمسترانگ\LTRfootnote{\lr{Armstrong}} حدس می‌زند که از لحاظ آواشناختی، پیروس یک «اسکله\LTRfootnote{\lr{Pier}}» بود. استیون با او مخالفت نمی‌کند و پیرو جواب آرمسترانگ می‌گوید که یک اسکله، «یک پل ناتمام» است. او خودش را تصور می‌کند که بعداً چاپلوسانه این لطیفه را برای خوشایندِ هینز تعریف می‌کند. متفکر درباره قتل پیروس و سزار، استیون به ناگزیریِ فلسفی برخی وقایع تاریخی می‌اندیشد - آیا تاریخ، به سرانجام رسیدن تنها حالت ممکنِ سلسله وقایع است یا یکی از حالات بیشمارِ آن می‌باشد؟

    استیون بحث کلاس را به سمت لیسیداسِ میلتون\LTRfootnote{\lr{Milton's \protect\textit{Lycidas}}} می‌برد و همچنان به تفکر درباره سوالات خودش درباره تاریخ ادامه می‌دهد، سوالاتی که هنگام خواندن ارسطو در کتابخانۀ پاریس به آنها فکر کرده بود. تصویری از شعر میلتون، استیون را به تفکر درباره تأثیر خدا روی همه آدمیان وا می‌دارد. استیون به خطوط یک معمای پیش پا افتاده فکر می‌کند و سپس تصمیم می‌گیرد به دانش‌آموزان که دارند وسائلشان را جمع می‌کنند که بروند در زمین هاکی بازی کنند، معمای خودش را بگوید. استیون در تنهایی به معمای حل‌نشدنی خودش درباره «روباهی که مادربزرگش را زیر یک بوته به خاک می‌سپارد\LTRfootnote{\lr{A fox burying his grandmother under a bush}}» می‌خندد.

    دانش‌آموزان کلاس را ترک می‌کنند به غیر از سارجنت\LTRfootnote{\lr{Sargent}} که به کمک در درس حساب نیاز دارد. استیون به سارجنتِ زشت نگاه می‌کند و عشق مادر سارجنت نسبت به خودش را تصور می‌کند. استیون به سارجنت، حاصلجمع‌ها را نشان می‌دهد، و کمی به لطیفۀ باک فکر می‌کند که می‌گفت تئوریِ هملتِ استیون را می‌توان با جبر اثبات کرد. با فکر دوباره به \lr{amor matris} یا عشق مادر، استیون خودش را به شکل کودکی به یاد می‌آورد که مانند سارجنت بدترکیب بود. سارجنت بیرون می‌رود تا به بازی هاکی بپیوندد. استیون بیرون می‌رود، سپس می‌رود تا در دفتر کار دیزی منتظر بماند در حالی که دیزی، مدیر مدرسه، در حال رفع و رجوع دعوای بچه‌ها سر بازی هاکی است.

    آقای دیزی، دستمزد استیون را پرداخت می‌کند و صندوق ذخیره‌اش را به رخ می‌کشد. دیزی برای استیون خطابه‎‌ای راجع به ارضا شدن با پولِ به دست آمده و اهمیت نگهداری دقیق پول و ذخیره کردن آن ایراد می‌کند. دیزی خاطرنشان می‌کند که بزرگترین افتخار یک انگلیسی این است که می‌تواند ادعا کند که هزینه‌هایش را خودش پرداخت کرده و هیچ بدهی ندارد. استیون قرض‌های فراوان خودش را ذهنی جمع می‌زند.

    دیزی می‌پندارد که استیون، که (دیزی) فکر می‌کند فنیان\LTRfootnote{\lr{Fenian}} (ملی‌گرای کاتولیک ایرلندی) است، به دیزی که توری\LTRfootnote{\lr{Tory}} (پروتستان وفادار به انگلستان) است بی‌احترامی می‌کند. دیزی درباره اعتبار ایرلندی‌اش بحث می‌کند - او شاهد اغلب ماجراهای ایرلند بوده است. دیزی سپس از استیون می‌خواهد که از نفوذش استفاده کند و نوشته‌ای از او را در روزنامه به چاپ برساند. در حالی که او دارد تایپ آن را به پایان می‌رساند، استیون نگاهی به تصاویر اسب‌های مسابقه در دفتر کار او می‌اندازد و یاد گردشی به پیست مسابقه همراه دوست قدیمی‌اش، کرانلی\LTRfootnote{\lr{Cranly}} می‌افتد.

    استیون فریادهایی را می‌شنود که به خاطر گلی در مسابقه هاکی سرداده شده است. دیزی، نوشتۀ تکمیل شده‌اش را به استیون می‌دهد و استیون آن را به سرعت می‌قاپد. نوشته، خطرات بیماری تب برفکی احشام را گوشزد می‌کند و اظهار می‌دارد که آن را می‌شود درمان نمود. به نظر می‌رسد که دیزی از تأثیر افرادی که در حال حاضر روی قضیه احاطه دارند متنفر است. همچنین به نظر می‌رسد که یهودیان را برای فساد مالی و نابودی اقتصاد ملی سرزنش می‌کند. استیون بحث می‌کند که بازرگانانِ حریص می‌توانند یهودی یا غیر یهودی باشند، ولی دیزی اصرار دارد که یهودی‌های نسبت به «نور\LTRfootnote{\lr{The light}}» گناه کرده‌اند.

    استیون، بازرگانان یهودی را که بیرونِ بازارِ بورسِ پاریس می‌ایستادند به خاطر می‌آورد. استیون دوباره با دیزی بحث می‌کند، و می‌پرسد چه کسی نسبت به نور گناه نکرده است. استیون، تعبیر دیزی از گذشته را رد می‌کند و می‌گوید: «تاریخ، کابوسی است که می‌خواهم از آن بیدار شوم». به طور طعنه آمیزی، همان موقع که دیزی دارد دربارۀ تاریخ به مثابه حرکت به سمت «هدف» جلوۀ خدا حرف می‌زند، گلی در بازی هاکی به هدف می‌نشیند\footnote{ گل (ورزش) و هدف در زبان انگلیسی هر دو معادل کلمه \lr{goal} هستند.}. استیون جواب می‌دهد که خدا چیزی بیش از «فریادی در خیابان» نیست. دیزی ابتدا بحث می‌کند که همه گناه کرده‌اند، سپس زنان را برای آوردن گناه به این دنیا سرزنش می‌کند. او فهرستی از زنان را بیان می‌کند که در طول تاریخ باعث نابودی و تباهی شده‌اند.

    دیزی پیش‌بینی می‌کند که استیون زیاد در مدرسه باقی نخواهد ماند، چرا که یک معلم بالفطره نیست. استیون می‌گوید که او بیشتر یک یادگیرنده است یا یاددهنده. استیون با بازگشت به موضوع نوشتۀ دیزی، خاتمه بحث را پیش می‌کشد. استیون سعی خواهد کرد که آن را در دو روزنامه به چاپ برساند. استیون از مدرسه بیرون می‌رود و به خوش‌خدمتی خودش نسبت به دیزی فکر می‌کند. دیزی به دنبال او می‌رود تا آخرین ضربه را به یهودی‌ها بزند - ایرلند هیچ وقت در حق یهودی‌ها جفا نکرده است چرا که آنها هیچ وقت اجازه ورود به کشور را نداشته‌اند.

    \chapter[پروتئوس]{پروتئوس\protect\footnote{\lr{Proteus}-\rl{در اسطوره‌های یونان، یکی از خدایان دریا که اشکال مختلف به خود می‌گرفته است. زمانی او را پسر پوزئیدون دانسته‌اند و زمانی ملازم و همنشین وی. او هم از قدرت پیشگویی برخوردار بود و هم از قدرت تغییر شكل در هر زمان كه اراده می كرد.}}}\label{ep:3}

    \chapter[کالوپسو]{کالوپسو\protect\footnote{\lr{Calypso}-\rl{در اسطوره‌های یونان، یک پری دریایی است. دختر اطلس بود و در جزیره اوگوگیا زندگی می‌کرد. اولیس در راه بازگشت از تروا، به این جزیره وارد شد. کالوپسو به او دل بست و او را هفت سال نزد خود نگاه داشت. به اولیس پیشنهاد کرد همیشه با او بماند و جاودان شود. اما اولیس در هوای خانه بود. عاقبت زئوس، هرمس را فرستاد تا کالوپسو را راضی کند دست از اولیس بردارد.}}}\label{ep:4}

    \chapter[لوتوفاگ‌ها]{لوتوفاگ‌ها\protect\footnote{\lr{Lotus-eaters} یا \lr{Lotophagi} یا \lr{Lotophaguses}-\rl{به معنی خورندگان نیلوفر آبی، در اسطوره‌های یونان، نام قبیله‌ای است که در ساحل لیبی زندگی می‌کردند. اولیس با همراهانش به جزیره لوتوفاگ‌ها وارد شد. آن‌ها از میوه نیلوفر آبی خوردند و حافظه خود را از دست دادند.}}}\label{ep:5}

    \chapter[هادس]{هادس\protect\footnote{\lr{Hades}-\rl{ در اساطیر یونانی، فرمانروای مردگان و دنیای زیرزمین، فرزند کرونوس و رئا است. او در قرعه‌کشی با برادرانش، بدترین سهم را برنده شد و آن جهان زیرین یا دنیای مردگان بود درصورتی که برادران او زئوس و پوزئیدون به ترتیب آسمان و دریا نصیبشان شد. از آنجایی که رعایای هادس را مردگان تشکیل می‌دادند، او به کسانی که موجب افزایش جمعیت سرزمینش می‌شدند بسیار علاقه داشت. مانند ارینی‌ها Erinnyes یا خشم و ناامیدی، که کارشان تعقیب گناهکاران و سوق دادن آن‌ها به سمت خودکشی بود.}}}\label{ep:6}

    \chapter[آیولوس]{آیولوس\protect\footnote{\lr{Aeolus}-\rl{ در اسطوره‌های یونان، پادشاه جزیره شناور آیولیا و خدای زمینی بادها است. زئوس قدرت مهار بادها را به او داده بود و خدایی زمینی محسوب می‌شد. با مهار باد به اولیس در رسیدن به تروا کمک کرد.}}}\label{ep:7}

    \chapter[لاستریگون‌ها]{لاستریگون‌ها\protect\footnote{\lr{Laestrygonians} یا \lr{Laestrygones} یا \lr{Laistrygones}-\rl{ در اساطیر یونان، قبیله‌ای از غول‌های آدمخوار. اولیس در راه بازگشت به ایتاکا با آنها مواجه شد. غول‌ها، بسیاری از مردان اولیس را خوردند و یازده کشتی از دوازه کشتی او را با پرتاب سنگ از بالای تپه نابود کردند.}}}\label{ep:8}

    \chapter[سیلا و کاریبد]{سیلا و کاریبد\protect\footnote{\lr{Scylla and Charybdis}-\rl{یا اسکیلا و کاریبدس یا سیلا و شاریبدیس، دو هیولای دریایی از اساطیر یونان هستند که توسط هومر مورد اشاره قرار گرفته اند. بعد از سنت‌های یونانی، محل آنها را در دو طرف تنگه مسینا و مقابل یکدیگر، در نظر گرفته‌اند. این محل بین سیسیل و سرزمین اصلی گراسیا مگنا یا همان یونان بزرگ (در جنوب ایتالیا) واقع است. گفته می‌شود که سیلا و کاریبد، در واقع در کنار یکدیگر، تهدیدی جدی و غیر قابل اجتناب در مسیر عبور ملوانان به حساب می‌آمدند؛ بدین ترتیب، کاریبد و سیلا هر دو در خود معنای جلوگیری کننده از عبور را دارند.}}}\label{ep:9}

    \chapter[صخره‌های سرگردان]{صخره‌های سرگردان\protect\footnote{\lr{Wandering Rocks} یا \lr{Planctae}-\rl{در اساطیر یونان، گروهی از صخره‌ها بودند که دریای بین آنها بیرحمانه خروشان بود.}}}\label{ep:10}

    \chapter[سیرن‌ها]{سیرن‌ها\protect\footnote{\lr{Sirens}-\rl{یا سایرن، یا حوری دریایی اساطیر یونان، گاهی به صورت موجودی با بدن یک پرنده و سر یک زن، و در سایر موارد به شکل تنها یک زن تصویر شده‌است. سیرن‌ها دختران خدای دریا فورکیس بوده‌اند، هرچند در نسخۀ دیگری از اساطیر، پدرشان خدای نهر، آکلوس دانسته شده است. آنها آوازی بسیار زیبا و فریبنده داشتند و دریانوردان را با آوای خود گمراه کرده و به کام صخره‌های مرگ‌آوری که بر روی آن آواز میخواندند، میکشیدند. اولیس، قهرمان افسانه‌ای یونان، توانست بدون هیچ خطری از جزیره آنان بگذرد، از آنرو که طبق نصیحت سیرسه ساحره، او از همراهانش خواست تا گوشهایشان را با موم پرکرده و او را محکم به دکل کشتی ببندند تا با اغوای آنان کشتی را به بیراهه نکشاند و بی هیچ خطری بتواند آواز آنان را بشنود.}}}\label{ep:11}

    \chapter[سیکلوپ]{سیکلوپ\protect\footnote{\lr{Cyclops}-\rl{یکی از موجودات افسانه‌ای در اساطیر یونانی است. سیکلوپ‌ها در اسطوره‌های یونان، غول‌هایی با یک چشم در وسط پیشانی هستند. آنها قدرتمند و سرسخت بودند. حرکات آنها همیشه همراه با خشونت و قدرت بود.}}}\label{ep:12}

    \chapter[ناوسیکائا]{ناوسیکائا\protect\footnote{\lr{Nausicaa}-\rl{در اسطوره‌های یونان، دختر آلکینوئوس است. زمانی که اولیس به جزیرۀ سخریا رسید او عاشق اولیس شد و از پدر خواست اجازه دهد با وی ازدواج کند. اما اولیس که قصد بازگشت به سرزمین خود را داشت نپذیرفت.}}}\label{ep:13}

    \chapter[گلۀ گاوِ خورشید]{گلۀ گاوِ خورشید\protect\footnote{\lr{Oxen of the Sun} یا \lr{Cattle of Helios}-\rl{در اساطیر یونان، در جزیرۀ تریناکیا می‌چریدند که معادل سیسیل امروزین است. هلیوس یا خدای خورشید، هفت گله گاو و هفت رمه گوسفند داشت.}}}\label{ep:14}

    \chapter[کیرکه]{کیرکه\protect\footnote{\lr{Circe}-\rl{در اسطوره‌های یونان، دختر هلیوس و پرسه است. جادوگری قدرتمند بود که دشمنانش را به حیوان تبدیل می‌کرد. پیکوس، سکولا و همراهان اولیس از جمله کسانی بودند که مورد خشم او قرار گرفتند.}}}\label{ep:15}

    \chapter[ائومایوس]{ائومایوس\protect\footnote{\lr{Eumaeus}-\rl{خوک‌چران اولیس. خدمتکار اولیس بود و او را در راه رسیدن به همسرش پنلوپه یاری داد. در اصل شاهزاده بود ولی کنیزی او را ربود و به لائرتس، پدر اولیس فروخت.}}}\label{ep:16}

    \chapter[ایتاکا]{ایتاکا\protect\footnote{\lr{Ithaca}-\rl{شهری که اولیس در آن زندگی می‌کرد.}}}\label{ep:17}

    \chapter[پنلوپه]{پنلوپه\protect\footnote{\lr{Penelope}-\rl{در اسطوره‌های یونان، همسر اولیس و مادر تلماخوس است. او در جوانی به دلیل زیبایی‌اش خواستگاران زیادی داشت. پدرش برای جلوگیری از دعوا و کشمکش مسابقه‌ای ترتیب داد تا برندۀ آن را به دامادی انتخاب کند. در این مسابقه اولیس سرافراز بیرون آمد و با پنلوپه ازدواج کرد.}}}\label{ep:18}

    \part{}
    \chapter{طرح کلی داستان}
    \noun{استیون ددالوس} در حال گذراندن ساعات اولیۀ صبح 16 ژوئن 1904 است و از دوست مسخره‌کننده‌اش، \noun{باک مالیگان} و \noun{هینز}، آشنای انگلیسی \noun{باک} دوری می‌جوید. وقتی \noun{استیون} می‌خواهد سر کار برود، \noun{باک} به او می‌گوید که کلید خانه را با خود نبرد و آنها را در ساعت 12:30 در میخانه ببیند. \noun{استیون} از \noun{باک} می‌رنجد.

    حدود ساعت 10 صبح، \noun{استیون} در کلاس درسش در مدرسۀ پسرانۀ \noun{گرت دیزی} دارد تاریخ درس می‌دهد. بعد از کلاس، \noun{استیون} با \noun{دیزی} ملاقات می‌کند تا حقوقش را بگیرد. \noun{دیزیِ} کوته‌فکر و متعصب، راجع به زندگی برای \noun{استیون} موعظه می‌کند. \noun{استیون} قبول می‌کند که نوشتۀ \noun{دیزی} دربارۀ بیماری احشام را به آشنایانش در روزنامه بدهد.

    \noun{استیون} بقیۀ صبحش را به تنها قدم زدن در ساحل \noun{سندی‌مونت} می‌گذراند و منتقدانه به خودِ جوان‌ترش و به ادراک و الهام فکر می‌کند. او در ذهنش شعری می‌گوید و آن را روی تکه کاغذی که از نوشتۀ \noun{دیزی} پاره کرده است می‌نویسد.

    در ساعت 8 صبح همان روز، \noun{لئوپلد بلوم} مشغول درست کردن صبحانه است و نامه و صبحانۀ زنش را برای او به رختخواب می‌برد. یکی از نامه‌های زنش از طرف مدیر تور کنسرتِ \noun{مالی}، \noun{بلیزس بویلان} است (\noun{بلوم} به این مظنون است که او عاشق زنش نیز هست) - \noun{بویلان} ساعت 4 بعدازظهر امروز قرار ملاقات دارد. \noun{بلوم} به طبقۀ پایین می‌رود و نامۀ دخترش \noun{میلی} را می‌خواند و سپس از خانه خارج می‌شود.

    در ساعت 10 صبح، \noun{بلوم} نامه‌ای عاشقانه از ادارۀ پست دریافت می‌کند - او با زنی به نام \noun{مارتا کلیفورد} تحت نام مستعار هنری \noun{فلاور} نامه‌نگاری می‌کند. او نامه را می‌خواند، مختصری در کلیسایی می‌ماند و سپس لوسیونِ \noun{مالی} را به داروخانه‌چی سفارش می‌دهد. او با \noun{بانتام لاینز} برخورد می‌کند که اشتباهاً فکر می‌کند \noun{بلوم} دارد به او راجع به اسب \noun{ثرواوی} در مسابقۀ بعدازظهر \noun{گلدکاپ} راهنمایی می‌کند.

    حوالی ساعت 11 صبح، \noun{بلوم} همراه با \noun{سایمون ددالوس} (پدر \noun{استیون})، \noun{مارتین کانینگهام} و \noun{جک پاور} به مراسم خاکسپاری \noun{پدی دیگنام} می‌رود. مردها با \noun{بلوم} مثل یک غریبه برخورد می‌کنند. در مراسم خاکسپاری، \noun{بلوم} به مرگ پسرش و پدرش فکر می‌کند.

    ظهر، \noun{بلوم} را در دفتر روزنامۀ \noun{فریمن} می‌بینیم که دارد دربارۀ یک آگهی برای \noun{کیزِ} مشروب‌فروش مذاکره می‌کند. چندین مردِ علاف منجمله \noun{مایلس کرافوردِ} ویراستار، در دفتر می‌چرخند و بحث‌های سیاسی می‌کنند. \noun{بلوم} برای حتمی‌کردن آگهی خارج می‌شود. \noun{استیون} با نامۀ \noun{دیزی} وارد دفتر روزنامه می‌شود. \noun{استیون} و بقیۀ مردها همزمان با بازگشت \noun{بلوم} دارند به سمت میخانه می‌روند. مذاکرات آگهی \noun{بلوم} توسط \noun{کرافورد} که دارد بیرون می‌رود، رد می‌شود.

    در ساعت 1 بعدازظهر، \noun{بلوم} با \noun{جوسی برین}، عشق قدیمی‌اش برخورد می‌کند و راجع به \noun{مینا پیورفوی} که در زایشگاه بستری است، صحبت می‌کنند. \noun{بلوم} وارد رستوران \noun{برتون} می‌شود ولی تصمیم می‌گیرد به سمت \noun{دیوی برن} برود تا نهاری سبک بخورد. \noun{بلوم} به یاد بعدازظهری عاشقانه با \noun{مالی} در \noun{هاوث} می‌افتد. \noun{بلوم} خارج شده و دارد به سمت کتابخانۀ ملی می‌رود که \noun{بویلان} را در خیابان می‌بیند و به داخل موزۀ ملی پناه می‌برد.

    در ساعت 2 بعدازظهر، \noun{استیون} دارد «تئوری هملت» خود را به طور غیررسمی برای \noun{اِی.ای} شاعر و \noun{جان اگلینتون}، \noun{بست} و \noun{لیستر} کتابدار توضیح می‌دهد. \noun{اِی.ای}، تئوری \noun{استیون} را سبک می‌شمارد و خارج می‌شود. \noun{باک} وارد می‌شود و با تمسخر، \noun{استیون} را به خاطر قال گذاشتن او و \noun{هینز} در میخانه سرزنش می‌کند. در راه خروجی، \noun{باک} و \noun{استیون} از کنار \noun{بلوم} می‌گذرند که آمده تا رونوشتی از آگهی \noun{کیز} بردارد.

    در ساعت 4 بعدازظهر، \noun{سایمون ددالوس}، \noun{بن دالرد}، \noun{لنه‌هان} و \noun{بلیزس بویلان} در نوشگاه هتل \noun{اورموند} گرد هم می‌آیند. \noun{بلوم} متوجه ماشین \noun{بویلان} در بیرون هتل می‌شود و تصمیم می‌گیرد او را زیر نظر بگیرد. \noun{بویلان} خیلی زود برای قرارش با \noun{مالی} خارج می‌شود و \noun{بلوم} با عصبانیت در رستوران \noun{اورموند} می‌نشیند - او موقتاً با آوازخوانی \noun{ددالوس} و \noun{دالرد} آرام می‌شود. \noun{بلوم} جواب نامۀ \noun{مارتا} را می‌نویسد و می‌رود که نامه را پست کند.

    در ساعت 5 بعدازظهر، \noun{بلوم} به میخانۀ \noun{بارنی کیرنان} می‌رود تا با \noun{مارتین کانینگهام} درباره مسائل \noun{مالی} خانوادۀ \noun{دیگنام} صحبت کند، ولی \noun{کانینگهام} هنوز نرسیده است. \noun{شهروند}، یک میهن‌پرست خشونت‌گرای ایرلندی، سیاه مست می‌شود و به یهودی بودنِ \noun{بلوم} می‌تازد. \noun{بلوم} جلوی \noun{شهروند} می‌ایستد و از صلح و عشق دربرابر خشونت بیگانه‌هراسانه دفاع می‌کند. \noun{بلوم} و \noun{شهروند}، قبل از اینکه کالسکۀ \noun{کانینگهام}، \noun{بلوم} را از صحنه دور کند، با هم در خیابان مشاجره می‌کنند.

    حوالی غروب آفتاب، \noun{بلوم} بعد از رفتن به خانۀ خانم \noun{دیگنام}، در ساحل \noun{سندی‌مونت} استراحت می‌کند. زنی جوان که \noun{گرتی مک‌داول} نام دارد متوجه می‌شود که \noun{بلوم} دارد از ساحل به او نگاه می‌کند. \noun{گرتی} عمداً پایش را بیشتر و بیشتر نشان \noun{بلوم} می‌دهد در حالیکه \noun{بلوم} دارد یواشکی استمنا می‌کند. \noun{گرتی} می‌رود و \noun{بلوم} چرت می‌زند.

    ساعت 10 شب، \noun{بلوم} به زایشگاه می‌رود تا به \noun{مینا پیورفوی} سر بزند. \noun{استیون} و چند نفر از دوستانش که دانشجوی پزشکی‌اند نیز در بیمارستان هستند و مشغول نوشیدن و وراجی با صدای بلند راجع به مسائل مرتبط با تولد هستند. \noun{بلوم} قبول می‌کند که به آنها ملحق شود، هر چند که درنهان به خاطر تقلای خانم \noun{پیورفوی} در طبقۀ بالا، با عیاشی آنها مخالف است. \noun{باک} از راه می‌رسد و مردها به میخانۀ \noun{بورک} می‌روند. موقع تعطیل شدن میخانه، \noun{استیون}، دوستش \noun{لینچ} را راضی می‌کند که به فاحشه‌خانه بروند و \noun{بلوم} آنها را تعقیب می‌کند تا مراقبشان باشد.
    \noun{بلوم} بالاخره، \noun{استیون} و \noun{لینچ} را در فاحشه‌خانۀ \noun{بلا کوهن} می‌یابد. \noun{استیون} مست است و فکر می‌کند که دارد روح مادرش را می‌بیند - مملو از خشم و دیوانگی، چراغی را با چوبدستی‌اش خرد می‌کند. \noun{بلوم} دنبال \noun{استیون} می‌رود و او را در بحث با یک سرباز انگلیسی که \noun{استیون} را کتک می‌زند، می‌یابد.

    \noun{بلوم}، \noun{استیون} را به هوش می‌آورد و او را به استراحتگاه رانندگان تاکسی می‌برد تا قهوه‌ای بخورد و سر حال بیاید. \noun{بلوم}، \noun{استیون} را به خانه‌اش دعوت می‌کند.
    بعد از نیمه‌شب، \noun{استیون} و \noun{بلوم} به خانۀ \noun{بلوم} می‌روند. آنها کاکائوی داغ می‌خورند و دربارۀ گذشته‌شان صحبت می‌کنند. \noun{بلوم} از \noun{استیون} می‌خواهد که شب را بماند. \noun{استیون} مودبانه تقاضای او را رد می‌کند. \noun{بلوم} او را بدرقه می‌کند و به داخل برمی‌گردد تا شواهد حضور \noun{بویلان} را پیدا کند. \noun{بلوم} هنوز حالش خوب است و به رختخواب می‌رود، و داستان روزش را برای \noun{مالی} تعریف کرده و از او می‌خواهد صبحانه‌اش را به رختخواب بیاورد.

    بعد از اینکه \noun{بلوم} خوابش می‌برد، \noun{مالی} بیدار می‌ماند و از تقاضای \noun{بلوم} برای آوردن صبحانه به رختخواب در تعجب است. ذهن او به دوران کودکی‌اش در \noun{جیبرالتر}، سکس بعدازظهرش با \noun{بویلان}، حرفۀ خوانندگی‌اش و \noun{استیون ددالوس} مشغول است. تفکراتش راجع به \noun{بلوم} در طی مونولوگی که با خود دارد به تندی تغییر می‌کند ولی در انتها با یادآوری لحظات عاشقانه‌ای که در \noun{هاوث} داشتند و با دیدی مثبت به پایان می‌رسد.
\end{document}